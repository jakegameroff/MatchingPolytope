%! TeX root: ../mpolytope.tex

\begin{frame}
\begin{center}
\Large \textbf{The Perfect Matching Polytope} 
\end{center}
\end{frame}

\begin{frame}
\frametitle{Preliminaries}
Let \( G = (V,E) \). We will work in the vector space \( \mathbb{R}^{E} \coloneqq \mathbb{R}^{|E|}  \).
\begin{itemize}
	\item Vectors in \( \mathbb{R}^{E}  \) have components indexed by \( E \).
	\item E.g. \( x = (x(e) : e \in E) \in \mathbb{R}^{E} \)
\end{itemize}
\vspace{0.3cm}
As such, each component of a vector in \( \mathbb{R}^{E}  \) contains information about an edge \( e \in E \).
\begin{itemize}
	\item E.g. If \( F \subseteq E \), define \( \chi_{F} \in \mathbb{R}^{E}  \) by \( \chi_{F}(e) = 1  \) if \( e \in M \) and \( \chi_{F} (e) = 0 \) otherwise.
\end{itemize}
\vspace{0.3cm}
\end{frame}

\begin{frame}
\frametitle{Preliminaries}
We must cover one final preliminary: \\
\vspace{0.3cm}
Given \( x \in \mathbb{R}^{E}  \) and \( F \subseteq E \), define \[ x(F) \coloneqq x \cdot \chi_{F} = \sum_{e \in F}^{} x(e) .  \]
\end{frame}

\begin{frame}
\frametitle{The Perfect Matching Polytope}
Let \( \mathcal{M}_{G}  \) denote the collection of perfect matchings of \( G \). Then, we define the \textbf{perfect matching polytope} \( \mathcal{P} \mathcal{M}  (G) \) of \( G \) by \[ \mathcal{P} \mathcal{M} (G) \coloneqq \operatorname{conv} (\{ \chi_{M} \in \mathbb{R}^{E} : M \in \mathcal{M} _{G}  \} ), \] where \( \operatorname{conv} (A) \) is the smallest convex set containing \( A \subseteq \mathbb{R}^{E}  \).
\end{frame}

\begin{frame}
\frametitle{The Perfect Matching Polytope}
The set \( \mathcal{P} \mathcal{M} (G) = \operatorname{conv} (\{ \chi_{M} \in \mathbb{R}^{E} : M \in \mathcal{M} _{G}  \} ) \) doesn't seem to help us investigate the perfect matchings in \( G \).
\begin{itemize}
	\item<2-> By definition, \( \mathcal{P} \mathcal{M} (G) \) is a polytope, so it would be really nice if we could \emph{find a linear program} whose optimum occurs at one of its corners. 
	\item<3> Let's do that now!
\end{itemize}
\end{frame}

\begin{frame}
\frametitle{The Perfect Matching Polytope}
\textbf{Observation 1:} \emph{If \( x \in \mathcal{P} \mathcal{M} (G) \), then \( x(e) \geq 0 \) for every \( e \in E \).}
\begin{proof}
	We may write \( x \) as a convex combination of characteristic vectors of perfect matchings in \( G \): \[ x = \lambda_1 \chi_{M_1} + \lambda_2 \chi_{M_2} + \cdots + \lambda_{n} \chi_{M_{n} } . \] Since \( \lambda_{j} \in [0, 1] \) and \( \chi_{M_{j} } \geq 0 \) for each \( j \in [n] \), it follows that \( x \) has non-negative components.
\end{proof}
\end{frame}

\begin{frame}
\frametitle{The Perfect Matching Polytope}
\textbf{Observation 2:} \emph{If \( x \in \mathcal{P} \mathcal{M} (G) \), then \( x(\delta (v)) = 1 \) for all \( v \in V \).}\\
(\emph{note: \( \delta (v) \) is the set of all edges containing \( v \) as an end})
\begin{proof}
Again, write \( x = \sum_{j=1}^{n} \lambda_{j} \chi_{M_{j} }  \) as a convex combination. Then
\begin{align*}
	x(\delta (v)) &= \sum_{e \in \delta (v)}^{} x(e) = \sum_{e \in \delta (v)}^{} \sum_{j=1}^{n} \lambda_{j} \chi_{M_{j} } (e) \\
		      &= \sum_{j=1}^{n} \sum_{e \in \delta (v)}^{} \lambda_{j} \chi_{M_{j} } (e) = \sum_{j=1}^{n} \lambda_{j} = 1,
\end{align*}
since each \( M_{j}  \) is a perfect matching.
\end{proof}
\end{frame}

\begin{frame}
\frametitle{The Perfect Matching Polytope}
\textbf{Observation 1:} \emph{If \( x \in \mathcal{P} \mathcal{M} (G) \), then \( x(e) \geq 0 \) for every \( e \in E \).} \\
\vspace{0.1cm}
\textbf{Observation 2:} \emph{If \( x \in \mathcal{P} \mathcal{M} (G) \), then \( x(\delta (v)) = 1 \) for all \( v \in V \).}\\
\vspace{0.5cm}
Observations 1 and 2 can be written compactly as the following: \\
\vspace{0.1cm}
\textbf{Observtion 3:} \emph{If \( x \in \mathcal{P} \mathcal{M} (G) \), then \( x \geq 0 \) and \( Ax = 1 \), where \( A = (a_{ve})_{v \in V, e \in E}  \) is the incidence matrix of \( G \).}
\begin{itemize}
	\item<2> \emph{This is really starting to look like a linear program...} 
\end{itemize}
\end{frame}

\begin{frame}
\frametitle{The Fractional Perfect Matching Polytope}
\textbf{Observtion 3:} \emph{If \( x \in \mathcal{P} \mathcal{M} (G) \), then \( x \geq 0 \) and \( Ax = 1 \), where \( A = (a_{ve})_{v \in V, e \in E}  \) is the incidence matrix of \( G \).}\\
\vspace{0.3cm}
Define the \textbf{fractional perfect matching polytope} \( \mathcal{F} \mathcal{P}\mathcal{M} (G)  \) of a graph \( G \) by \[ \mathcal{FPM}(G) \coloneqq \{ x \in \mathbb{R}^{E} : x \geq 0 \mbox{ and } Ax = 1 \}.   \] Then, it follows from observation 3 that \( \mathcal{PM} (G) \subseteq \mathcal{FPM} (G) \).
\begin{itemize}
	\item<2> \emph{This raises a natural question...} 
	
\end{itemize}
\end{frame}

\begin{frame}
\begin{center}
\Large Are there any graphs \( G \) with \( \mathcal{PM} (G) = \mathcal{FPM} (G) \)?
\end{center}
\end{frame}

\begin{frame}
\begin{center}
	\Large Yes!
\end{center}
\end{frame}

\begin{frame}
\begin{center}
	\Large If \( G \) is \textbf{\alert{bipartite}}, then \( \mathcal{PM} (G) = \mathcal{FPM} (G) \).
\end{center}
\end{frame}

\begin{frame}
\textbf{Lemma 1:} \emph{Let \( A \) be the incidence matrix of a bipartite graph \( G \). Then \( A \) is totally unimodular; that is, every square submatrix \( B \) of \( A \) satisfies \( \det B \in \{ 0, \pm 1 \}  \).}
\begin{proof}
Let \( B \) be an \( m \times m \) submatrix of \( A \). The proof is by induction on \( m \). If \( m = 1 \) then clearly \( \det B \in \{ 0,1 \}  \), so fix \( m \geq 2 \).
\begin{itemize}
	\item If \( B \) has a column with at most one non-zero entry, then we may cofactor-expand along this column and use the IH to obtain \( \det B \in \{ 0, \pm 1 \}  \).
	\item Otherwise, every column of \( B \) has exactly two non-zero entries. Since any two non-zero entries from the same column are in different partite sets, the sum of all rows pertaining to vertices from one partite set equals the sum of all rows from the other. So \( \det B = 0 \) since its rows are linearly dependent.
	
\end{itemize}
\end{proof}
\end{frame}

\begin{frame}
	\textbf{Lemma 2:} \emph{If \( A \) is totally unimodular and \( b \) is a vector with integer components, then all corners of \( P = \{ x \geq 0 : Ax \leq b \}  \) have integer components.}
\begin{proof}
	The proof is long and somewhat off-topic. Given the time constraint, see section \( n \) \alert{here}.
\end{proof}
\end{frame}

\begin{frame}
	\textbf{Theorem:} \emph{If \( G \) is a bipartite graph, then \( \mathcal{PM} (G) = \mathcal{FPM} (G) \).}
\begin{proof}
We have already seen that \( \mathcal{PM} (G) \subseteq \mathcal{FPM} (G) \). From lemma 1, the incidence matrix \( A \) of \( G \) is totally unimodular; and so lemma 2 implies that every corner of \( \mathcal{FPM} (G) \) has integer components. 
\end{proof}
\end{frame}

