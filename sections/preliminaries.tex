%! TeX root: ../mpolytope.tex

\begin{frame}
\frametitle{Agenda} \textbf{Plan for today:} 
\begin{itemize}
	\item<2-> Quick review of linear programming \& polytopes
	\item<3-> Fundamentals of matching theory
	\item<4-> The perfect matching polytope
		\begin{itemize}
			\item<5-> Bipartite \& general graphs
			\item<6-> Application to cubic graphs
		\end{itemize}
\end{itemize}
\end{frame}

\begin{frame}
\begin{center}
	\Large \textbf{Linear Programming \& Polytopes}
\end{center}
\end{frame}

\begin{frame}
\frametitle{Linear Programming}
The goal of \textbf{linear programming} is to maximize (or minimize) a linear objective function subject to a collection of linear constraints. \\
\vspace{0.3cm}
\pause
For example:
\begin{align*}
\text{maximize} \quad &z = 5x_1 + 3x_2 - 7x_3 \\
\text{subject to} \quad & x_1 + x_2 + x_3 \leq 12 \\
& 4x_1 + 5x_3 \leq 50 \\
& x_1, x_2, x_3 \geq 0
\end{align*}
\end{frame}

\begin{frame}
We can express a linear program (LP) compactly using matrices:
\pause
\begin{itemize}
	\item<2-> Goal: Find a vector \( x \) such that \( c^{T} x \) is maximized and \( x \) satisfies the constraints \( A x \leq b \) and \( x \geq 0 \).
	\item<3> (\( x, c \in \mathbb{R}^{n}  \), \( A \in \mathbb{R}^{m \times n} \), and \( b \in \mathbb{R}^{m}  \))
\end{itemize}
\end{frame}

\begin{frame}
    \frametitle{The Dual of an LP}

    Given a linear program in standard form:
    \begin{align*}
        \text{maximize} \quad & c^T x \\
        \text{subject to} \quad & Ax \leq b \\
                                & x \geq 0.
    \end{align*}
    \pause
    The dual LP is formulated as:
    \begin{align*}
        \text{minimize} \quad & b^T y \\
        \text{subject to} \quad & A^T y \geq c \\
                                & y \geq 0.
    \end{align*}
\end{frame}

\begin{frame}
\frametitle{The Dual of an LP}
The primal and dual LPs provide bounds on each other’s optimal values. If an LP has an optimum at \(x = \hat{x}  \), then its dual also has this optimum (Strong duality theorem).
\end{frame}

\begin{frame}
\frametitle{Polytopes}
A \textbf{polytope} is the convex hull of a finite collection of vectors.
\end{frame}

\begin{frame}
\frametitle{Polytopes}
\begin{figure}
        \centering
        \begin{minipage}{0.32\textwidth}
            \centering
            \includegraphics[width=\linewidth]{/Users/jakeg/Desktop/Matching Polytope/images/convexhull.png}
        \end{minipage}\hfill\pause
        \begin{minipage}{0.32\textwidth}
            \centering
            \includegraphics[width=\linewidth]{/Users/jakeg/Desktop/Matching Polytope/images/convex-polytope.png}
        \end{minipage}\hfill\pause
        \begin{minipage}{0.32\textwidth}
            \centering
            \includegraphics[width=\linewidth]{/Users/jakeg/Desktop/Matching Polytope/images/600cell.png}
        \end{minipage}
\end{figure}
\end{frame}

\begin{frame}
\begin{center}
	{\Large How are polytopes related to LP?} \\
	\vspace{0.3cm}
	\pause
	\emph{via an equivalent definition...} 
\end{center}
\end{frame}

\begin{frame}
\frametitle{Polytopes \& LP}
A vector \( \mathbf{x} \) is called a \textbf{feasible solution} if it satisfies the linear constraints of the LP, i.e. if \( A \mathbf{x} \leq \mathbf{b} \) and \( \mathbf{x} \geq \mathbf{0} \).\\
\pause
\vspace{0.3cm} The set \( P = \{ \mathbf{x} \geq \mathbf{0} : A\mathbf{x} \leq \mathbf{b} \}  \) of all feasible solutions is called a \textbf{polyhedron}. 
\begin{itemize}
	\item<3-> Bounded polyhedra are called \textbf{polytopes}.
	\item<4-> If the set of all feasible solutions to an LP is a polytope, then one of its corners is an optimum.
\end{itemize}
\end{frame}

\begin{frame}
\frametitle{The Big Picture}
We would like to solve combinatorial optimization problems using algorithms.\\
\pause
\vspace{0.3cm}
Some examples:
\pause
\begin{itemize}
	\item Finding min-weight edge or vertex covers in graphs.\pause
	\item Finding (pure, mixed, correlated) Nash equilibria in games. \pause
	\item Finding max flows and min cuts in flow networks. \pause
	\item Finding min-weight perfect matchings in graphs. \pause
\end{itemize}
\vspace{0.3cm}
If we can reduce these problems to solving linear programs, we can leverage efficient LP solvers to obtain optimal solutions.
\end{frame}

\begin{frame}
\begin{center}
\Large This talk is about the connection between linear programming, polytopes, and perfect matchings.
\end{center}
\end{frame}

\begin{frame}
\frametitle{Perfect Matchings}
A \textbf{matching} in a simple graph \( G = (V,E) \) is a subset \( M \subseteq E \) of edges such that no two edges in \( M \) share an end. \pause So every vertex \( v \in V \) is incident to \emph{at most} one edge in \( M \). \\
\pause
\vspace{0.3cm}

A matching \( M \) is \textbf{perfect} if every vertex \( v \in V \) is incident to \emph{exactly} one edge in \( M \). 
\end{frame}

\begin{frame}
\frametitle{Perfect Matchings}
A perfect matching (red edges) in the Petersen graph:
\begin{center}
\includegraphics[scale=0.1]{/Users/jakeg/Desktop/Matching Polytope/images/peter-match.png}
\end{center}
\end{frame}
